\subsection{Actors} \label{ssc:actors}
The actors in the application domain include the employees and admins. The differentiation of the two is made, because the admin has access to specific functionality and pages within the system, which are not available to the employee. As mentioned in \autoref{ch:problemdomain}, the admin is a role which an employee can take. This means that an admin has the same abilities as the employee as well as the extra functionalities the role provides.
\par
The descriptions of the actors include their level of experience with similar systems, their goals for using the system, and examples of the actor.

\fancyLayout{actor}{Admin}
    {Specification of the \textit{admin} actor.}
    {actor:admin}
    {
        \textbf{Goal:} The admin manages the system. The admin's basic goal is to keep track of the company assets.
        \vskip 0.2cm
        
        \textbf{Characteristics:} The admin is the primary user of the system and is the only actor who can change anything regarding the assets in the system. They are experienced with using enterprise systems \citep{EnterpriseSystems} and handling the assets of their department.
        \vskip 0.2cm
        
        \textbf{Example:} This admin is used to organising large quantities of assets, and sees the asset management system as a supplement to their organisational tools.
    }

The admin is, as mentioned, the primary user of the system and accommodating their needs has been the top priority, when it comes to accommodating actor needs.

\fancyLayout{actor}{Employee}
    {Specification of the \textit{employee} actor.}
    {actor:employee}
    {
        \textbf{Goal:} The \textit{Employee} can borrow the company assets. They have a need for borrowing assets that are relevant to their work.
        \vskip 0.2cm
        
        \textbf{Characteristics:} The \textit{Employee} is the secondary user of the system. They can only borrow, comment on, and view assets within the system. They have varying degrees of technical competence.
        \vskip 0.2cm
        
        \textbf{Examples:} \textit{Employee A} is familiar with computer systems and has no trouble finding their way around the asset management system.
        
        \vskip 0.1cm
        
        \textit{Employee B} is less experienced with computer systems and is more comfortable talking directly to the admin instead of using the system themselves. 
    }
    
As a secondary user of the system, the needs of the employee will have less priority than the needs of the admin.
\newpage