\section{Functions}\label{sc:functions}

This section will present the functions in the system. Table \ref{tab:functions} shows all the top-level functions that the system will be able to perform. All the tables describing functions have columns with the name of the function, the complexity and the type of function.

\begin{table}[H]
\centering
%\resizebox{\textwidth}{!}{%
    \begin{tabular}{|l|l|l|}
        \hline
        \textbf{Function} & \textbf{Complexity} & \textbf{Type} \\
        \hline
        \hline
        Add Asset & Complex & Update\\
        \hline
        Remove Asset & Medium & Update\\
        \hline
        Edit Asset & Complex & Update\\
        \hline
        Search Asset & Medium & Compute/Read\\
        \hline
        View Asset & Simple & Read\\
        \hline
        Add Template & Complex & Update\\
        \hline
        Remove Template & Medium & Update\\
        \hline
        Edit Template & Complex & Update\\
        \hline
        Add Tag & Simple & Update\\
        \hline
        Remove Tag & Medium & Update\\
        \hline
        Rename Tag & Simple & Update\\
        \hline
        Tag Asset & Simple & Update\\
        \hline
        Untag Asset & Simple & Update\\
        \hline
        Authenticate User & Simple & Read\\
        \hline
        Check access level & Simple & Read\\
        \hline
        Export report & Complex & Compute/Read\\
        \hline
    
    \end{tabular}
%}
\caption{Table showing all the different top-level functions as well as their complexity and type.}\label{tab:functions}
\end{table}

\par

Table \ref{tab:functions} contains several functions rated complex. These functions have been divided into several smaller functions, with a lower complexity.
% Beskriver hvorfor det er gjort, ved ikke om det skal med, men det er god fyldertekst
This is done in order to divide the complex problems, into smaller problems, which are easier to implement. 
 
\begin{center}
    \textbf{Adding an asset}
\end{center}

% Add asset table
\begin{table}[H]
\centering
%\resizebox{\textwidth}{!}
\caption{Table showing the different functions involved in adding an asset along with their complexity and type.}\label{tab:AddAssetFunctions}
\end{table}

\par

\begin{center}
    \textbf{Editing an asset}
\end{center}

% Edit Asset Table
\begin{table}[H]
\centering
%\resizebox{\textwidth}{!}
\caption{Table showing the different functions involved in editing an asset along with their complexity and type.}\label{tab:EditAssetFunctions}
\end{table}

\par

\begin{center}
    \textbf{Adding a template}
\end{center}

% Add template table 
\begin{table}[H]
\centering
%\resizebox{\textwidth}{!}
\caption{Table showing the different functions involved in adding a template along with their complexity and type.}\label{tab:AddTemplateFunctions}
\end{table}

\par

\begin{center}
    \textbf{Editing a template}
\end{center}

% Edit template table
\begin{table}[H]
\centering
%\resizebox{\textwidth}{!}
\caption{Table showing the different functions involved in editing a template along with their complexity and type.}\label{tab:EditTemplateFunctions}
\end{table}

\par

\begin{center}
    \textbf{Exporting a report}
\end{center}

% Export report table
\begin{table}[H]
\centering
%\resizebox{\textwidth}{!}
\caption{Table showing the different functions involved in exporting a report along with their complexity and type.}\label{tab:ExportReportFunctions}
\end{table}