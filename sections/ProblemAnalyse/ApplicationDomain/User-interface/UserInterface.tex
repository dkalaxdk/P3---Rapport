\subsection{User interface}\label{ssc:UIAnalysis}
% goal and target users
The goal of the user interface is to provide an overview of the assets managed by the system, and enable the user to oversee them. The target users for this system are experienced in using computer programs. It is therefore not necessary to guide the user thoroughly through the systems functionalities.
\par

% supported activities and contexts. Short bursts of usage
Setting up the system and adding all the assets and relevant information will take time and effort on the users part. Afterwards, however, the user will only interact with the system in small bursts, when making small additions or deletions. There will not be an employee assigned to the system full time. Because of this, it is relevant to design the system in a way that allows the user to perform their tasks as fast as possible, and get on with their other work.
\par
As the work periods are short, user attention is less of an issue, since the user should be able to plan their work in the system to minimize external distractions. As the users work at an IT department, it is however possible for unpredictable interruptions to occur, when other employees might come by with questions or requests. As such, the system should be designed in a way that allows for the users attention to be divided, at least to a certain extent, without the user making major mistakes. This can be done by making it possible to save what is being worked on before it is completely done, so the user can return to finish it later. Making the system fast and responsive can also make it easier to handle distractions, as it will be easier to quickly finish the current task, despite potential interruptions. 
\par

% Hvordan vil vi designe UI'et?
The user interface should only contain the utmost required interface-elements. As mentioned in \autoref{ch:introduction} and \autoref{ch:problemdefinition}, the client needs an easier process for managing assets compared to existing solutions on the market. Existing solutions often present the user with a number of fields depending on the category an asset belongs to (see \autoref{fig:too-many-fields}). This often results in empty and unnecessary fields that are never used. 

\begin{figure}[H]
    \centering
    \frame{\includegraphics[width=0.9\textwidth]{figures/other-systems/snipeitapp-create-asset-ui-screenshot.png}}
    \caption{Example of an asset management system with too many fields that need to be filled out. \cite{SnipeIT}}
    \label{fig:too-many-fields}
\end{figure}

% Måske GUI vs. Command Prompt? Måske en pointe i at TUI er simplere?
Since the client is experienced computer users, it is relevant to consider what kind of user interface the system should be implemented. A simple option is to use a text-based user interface (TUI), as the employees at the IT department is expected to know how to use a terminal to access the system. The client, however, also wanted all other employees at the zoo to be able to access the system, in order to view the assets, and it is possible that the other departments also will be using the system. Since it cannot be expected that employees at other departments are equally qualified to use a TUI, a graphic user interface (GUI) has been chosen. 
\par
Below are two images of a graphical user interfaces that comply with the clients wishes (see \autoref{fig:add_asset_no_tags} and \autoref{fig:add_asset_with_tags}). These images will form a basis for the final user interface design.
\par
The images show the process of how an asset is added to the system, and how fields are attached to the asset through tags.

\begin{figure}[H]
    \centering
    \frame{\includegraphics[width=0.9\textwidth]{figures/wireframes/add-asset-no-tags.png}}
    \caption{The process of creating a new asset, at this point no tags or custom fields were added, resulting in no fields beside base fields.}
    \label{fig:add_asset_no_tags}
\end{figure}

\begin{figure}[H]
    \centering
    \frame{\includegraphics[width=0.9\textwidth]{figures/wireframes/add-asset-with-tags.png}}
    \caption{An example of how the user interface for adding a new asset looks after adding some tags and custom fields.}
    \label{fig:add_asset_with_tags}
\end{figure}

There are two types of users of the system, the administrators (the admin class) of the system and the other employees (the employee class), who have much more restricted access to the system. The images above (\autoref{fig:add_asset_no_tags} and \autoref{fig:add_asset_with_tags}) show the user interface for the admin. If an employee user accesses the system, the interface will be limited with fewer functionalities, as these users are not allowed to make changes to the systems content.