\subsection{PACT analysis}\label{ssc:PACT}
A PACT analysis \cite[chap 2]{DEB} has been performed to better understand the interactions between the users and the system. The analysis gives a better description of the users, their needs and restrictions, the context in which the system is used, the activities surrounding the system, and the technologies involved.\\

\textbf{People}:\\
%(Physiological differences, Psychological differences, Social differences, Domain Expertise)
It is important to analyze the interactions with the system, to ensure ease of use for the people using the system. It should be taken into consideration, how these people are different from each other and what special needs they might have. The people who will interact with the system are:

\begin{itemize}
    \setlength\itemsep{0.05em}
    \item Admins at the zoo (primarily at the IT department)
    \item Employees at the zoo
\end{itemize}

The presented people have no psychological, physiological, or social differences, that came up during the interviews, relevant to interacting with the system. The representatives from the IT department, which means they are used to working with computers. They are the main users of the system, as they are the ones who will use it the most. However, they have requested the interface to be minimalistic and easy to use. The other employees at the zoo are expected to have experience with computer-based systems through their work, but not as deep of an understanding as an employee of the IT department. This information has been derived from the interviews (see \autoref{sc:aalborgZoo}).\\

\textbf{Activities}: \\
%(Purpose of activities to be supported by the system, Temporal aspects, Collaboration, Complexity, Safety criticality, Nature of system content) \\
The system will primarily be used for keeping track of and exporting lists of assets. The users will keep track of the assets by adding them to, editing them in, and removing them from the system. They will also export a list of changes made to the assets (see \autoref{sc:aalborgZoo}). On occasion they may also want a specific asset and will search through the database. Some users will only use the system to see assets and their properties, and comment on them.\\

\textbf{Context}: \\
%(Physical, Social, Organizational) \\
The system will be used at Aalborg Zoo, primarily on desktops and laptops located within the zoo. The laptops might be moved around the premise. The social context at the zoo is not expected to be affected by the system, as it does not contain any elements geared towards any social aspects.\\
 
\textbf{Technology}:
%(That could support users in the domain)
The relevant technologies related to the system and its implementation include the following:

\begin{itemize}
    \setlength\itemsep{0.05em}
    \item Computer
    \item Screen
    \item Barcode scanner
    \item Mouse
    \item Keyboard
    %\item Database
    %\item Internet
\end{itemize}

The computer will be used to run the system and the screen will be used as an output device, while the barcode scanner, mouse, and keyboard will be used for input. % The database will function as storage for everything added to the system, such as assets, and the internet will be used in the context of connecting to the database and maintaining the model across all connected devices running the system.
\\\\
With an understanding of the people, activities, contexts, and technologies surrounding the system, the user interface can be designed.
\newpage