\section{Function activity}\label{sc:function}
Based on the use cases, it is possible to determine the functions necessary for the system to adequately fulfill the requirements. These functions have been classified by their type and complexity in the following function list. The complex functions have then been further split into smaller functions to explain their processes. This is done following the functions analysis and the complexity definitions from \cite[chap 7]{OOAD}. The complexity definitions are:
\begin{itemize}
    \item Simple: Sets or reads the value of an attribute in an existing object.
    \item Medium: Creates a new object and connects it by object structure(s) to other objects.
    \item Complex: Reads from or creates several objects (from different classes).
\end{itemize}
With these definitions, the following function table has been constructed.

\begin{table}[H]
\centering
    \begin{tabular}{|l|l|l|}
        \hline
        \textbf{Function} & \textbf{Complexity} & \textbf{Type} \\
        \hline
        \hline
        Add asset & Complex & Update\\
        \hline
        Remove asset & Simple & Update\\
        \hline
        Update asset information & Complex & Update\\
        \hline
        Comment asset & Medium & Update\\
        \hline
        Search for asset & Medium & Compute/Read\\
        \hline
        View asset & Simple & Read\\
        \hline
        Add department & Simple & Update\\
        \hline
        Remove department & Simple & Update\\
        \hline
        Authenticate user & Simple & Read\\
        \hline
        Check access level & Simple & Read\\
        \hline
        Export report & Complex & Compute/Read\\
        \hline
        Attach tag & Simple & Update\\
        \hline
        Detach tag & Simple & Update\\
        \hline
    \end{tabular}
\caption{Function table showing all the different top-level functions as well as their complexity and type.}\label{tab:functions}
\end{table}

The function table (see \autoref{tab:functions}) contains multiple complex functions. As mentioned earlier, these functions have been further split into smaller functions to better understand the operations.


\fancyLayout
    {function}
    {Add asset}
    {The table shows the different functions involved in adding an asset as well as their complexity.}
    {function:add_asset}
    {
        \centering
        \begin{tabular}{|l|l|l|}
            \hline
            \textbf{Function} & \textbf{Complexity} & \textbf{Type}\\
            \hline
            \hline
            Add relevant information about the asset & Simple & Update \\
            \hline
            Save Asset to model & Medium & Update \\
            \hline
            Update connections to tags & Medium & Update\\
            \hline
        \end{tabular}
}

The \textit{Add asset} function consists of three smaller functions, as seen in \autoref{function:add_asset}. When adding an asset, first the admin enters the relevant information, which is then saved with the asset to the model. Lastly, the \textit{Asset-tag relation}s connected to the asset are updated. This last step possibly creates and deletes \textit{Asset-tag relation}s between the asset and the relevant tags.

\fancyLayout
    {function}
    {Update asset information}
    {The table shows the different functions involved in updating an assets information.}
    {function:update_asset_information}
    {
        \centering
        \begin{tabular}{|l|l|l|}
            \hline
            \textbf{Function} & \textbf{Complexity} & \textbf{Type}\\
            \hline
            \hline
            Load asset from model & Simple & Read \\
            \hline
            Update relevant information about the asset & Simple & Update \\
            \hline
            Save asset to model & Medium & Update \\
            \hline
            Update connections to tags & Medium & Update\\
            \hline
        \end{tabular}
}

The \textit{update asset information} function is made up of the smaller functions, as shown in \autoref{function:update_asset_information}, which will be executed in the shown order. First, the asset is loaded from the model and formatted as an asset in the system. Then the changes to the information is applied to the asset. The asset is then saved to the model with the changes and its connections to tags are updated. Just as the \textit{Add asset} function, the last step possibly creates and deletes \textit{Asset-tag relation}s between the asset and the relevant tags.

\fancyLayout
    {function}
    {Export report}
    {The table shows the different functions involved in exporting a report.}
    {function:export_report}
    {
        \centering
        \begin{tabular}{|l|l|l|}
            \hline
            \textbf{Function} & \textbf{Complexity} & \textbf{Type}\\
            \hline
            \hline
            Load assets to be included & Simple & Update \\
            \hline
            Create entries & Simple & Update \\
            \hline
            Build report & Simple & Update \\
            \hline
            Export to CSV-format & Medium & Compute \\
            \hline
        \end{tabular}
}

The \textit{export report} function involves the four smaller functions listed in \autoref{function:export_report}, happening in order. The first step is that the selected assets are loaded. Then report entries are created based on the assets, which are then added to the report. Finally the report is exported to the admin's computer formatted as a Comma Separated Values (CSV) file.
\par
This concludes the analysis of functions. The understanding of the users and functions has been used in the following section as a basis for the interface.