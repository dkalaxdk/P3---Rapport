\section{Behavior} \label{sc:behavoir}
In the following section, the behavior of each class will be examined. A deeper understanding of these behaviours will be achieved through state chart diagrams.
\\\\

\large{\textbf{Asset}}
\begin{figure}[H]
    \centering
    \includegraphics[width=1\textwidth]{figures/StateCharts/Asset_state_chart.png}
    \caption{State chart diagram for the \textbf{Asset} class}
    \label{fig:asset_statechart}
\end{figure}

An \textit{Asset} object is created when the asset is acquired. At first, its state is \textbf{available}, and when the asset is loaned out to an employee, it changes state to \textbf{loaned out}. From any of these two states, the asset can be disposed of and no longer exist within the problem domain. Being loaned out will, in the system, simply consist of the asset being tagged with an employee and the return of the asset will simple be removing this relation to the tag of that employee. The information on the asset, not including the tags attached to it, an be updated at all times during its life cycle in the system, and can occur multiple times.
\\\\

\large{\textbf{Employee}}
\begin{figure}[H]
    \centering
    \includegraphics[width=1\textwidth]{figures/StateCharts/StateChart_Employee.png}
    \caption{State chart diagram for the \textbf{Employee} class}
    \label{fig:employee_statechart}
\end{figure}

An \textit{Employee} object is created when a person is hired at the zoo and gains the status employed. An employee can potentially be granted admin access, which can later be revoked as well. An employee can also borrow an asset and return it later. The employee object seizes to exist when the person is fired. The event of returning an asset can only happen when the employee has borrowed the given asset, but an employee can borrow multiple assets at a time, hence both events are iterations.\\
The employee can be granted admin access multiple times during the employment but of course, the access an only be revoked after the access has been granted. Both thing can happen multiple times and therefore are iterations. These two events also mark the initialization and end of an admin.
\\\\

\large{\textbf{Admin}}
\begin{figure}[H]
    \centering
    \includegraphics[width=0.8\textwidth]{figures/StateCharts/Admin_state_chart.png}
    \caption{State chart diagram for the \textbf{Admin} class}
    \label{fig:admin_statechart}
\end{figure}

The \textit{Admin} object is created when an employee is granted admin access. As an admin, the employee is responsible for loaning out assets to other employees and receiving the assets when they are returned. In the system, the admin is also responsible of adding tags, assets, activating departments, and maintaining these elements by updating and removing them. The admin object is terminated when the employee's admin access is revoked.
\\\\

\large{\textbf{Tag}}
\begin{figure}[H]
    \centering
    \includegraphics[width=0.8\textwidth]{figures/StateCharts/Tag_state_chart.png}
    \caption{State chart diagram for the \textbf{Tag} class}
    \label{fig:loan_statechart}
\end{figure}

A \textit{Tag} object is created by an admin and can be attached/detached to/from assets multiple times during its life cycle. The tag is destroyed when an admin deletes it from the system.
\par

With the behaviours of the classes described, the problem domain has been analysed. The following section will sum up the results of the analysis.

% \large{\textbf{Loan}}
% \begin{figure}[H]
%     \centering
%     \includegraphics[width=0.8\textwidth]{figures/StateChart_Loan.png}
%     \caption{State chart diagram for the \textbf{Loan} class}
%     \label{fig:loan_statechart}
% \end{figure}

% A \textbf{Loan} object is created when an asset is loaned out and seizes to exist when the asset is returned.
% \newline\par
% With the behaviours of the classes described, the problem domain has been analysed. The following section will sum up the results of the analysis.