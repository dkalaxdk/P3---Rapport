\subsection{Classes} \label{ssc:classes}
% En beskrivelse af de klasser der er i problem domænet
% Asset, Admin, Employee, Department, Location

Based on the system definition, the classes \textit{Asset}, \textit{Employee}, \textit{Admin}, and \textit{Department} have been defined.
% During the interviews, Aalborg Zoo also mentioned a potential organisational structure to ease the use of the system. This structure is based on tags, which can be attached to assets and used to filter and add details to assets. The structure adds a \textit{Tag} class to the system.
% \par

\textbf{Asset}\\
The \textit{Asset} class represents assets in the problem domain and contains all relevant information of the asset.
\par

\textbf{Employee}\\
The \textit{Employee} class represents an employee at the zoo. Assets can be loaned out to an \textit{Employee} by an \textit{Admin}.
\par

\textbf{Admin}\\
The \textit{Admin} class represents an administrator, who is an employee with more responsibilities. The \textit{Admin} manages the assets, as well as which \textit{Employees} have loaned which assets.
\newpage

\textbf{Department}\\
The \textit{Department} class reflects a department within the zoo. These departments each have their own set of assets. 
\\\\
These four classes interact with each other through events, which will be addressed in \autoref{ssc:events}.
\par
The \textit{Admin} and \textit{Employee} classes are also represented in the application domain as actors in context of the system.


