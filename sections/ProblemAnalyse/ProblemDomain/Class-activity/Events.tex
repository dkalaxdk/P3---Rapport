\subsection{Events}\label{ssc:events}
% Beskriv de events der kan forekomme i problemdomænet
The interaction of the classes defined in \autoref{sc:classes} can be described as events. These events will be defined and described in the following section, this is done to get a better understanding of the relation between the classes in the system. \\\\
%The classes, defined in \autoref{sc:classes}, interact with each other and the system model. These interactions can be described as events, the events will now be named and described.

\textbf{Asset acquired:}\\
\textit{Asset acquired} is the event occurring, when the zoo or a department acquire a new asset. The event involves the acquired asset and the related department.\\

\textbf{Asset disposed of:}\\
\textit{Asset disposed of} happens when an asset is either disposed of by an admin or lost or broken by an employee. This event affects the disposed asset and the related department. \\

\textbf{Asset loaned out:}\\
\textit{Asset loaned out} is an event occurring whenever an admin loans out an asset to an employee. It involves the asset, the employee to whom the asset is loaned, and an admin creating the loan.\\

\textbf{Asset returned:}\\
\textit{Asset returned} is an event occurring whenever an admin registers the return of an asset loaned out to an employee. This event involves the same classes as \textit{asset loaned out}, as it is just the reversed event.\\

\textbf{Employee hired:}\\
\textit{Employee hired} happens whenever an employee is hired into the company. \\

\textbf{Employee fired:}\\
\textit{Employee fired} is the event of an employee being fired or quitting.\\

\textbf{Admin access gained:}\\
\textit{Admin access gained} occurs when an employees access is raised to admin level. This means further functionality is available to the employee within the system.\\

\textbf{Admin access revoked:}\\
\textit{Admin access revoked} occurs when an admins access is lowered to standard employee capabilities. It is possible to raise the access level to admin again afterwards. \\

% \textbf{Department activated:}\\
% \textit{Department activated} occurs when a department is created/activated and is available for assets. Assets cannot exist without an active department. \\

% \textbf{Department deactivated:}\\
% \textit{Department deactivated} occurs when a department is suspended/deactivated/closed or in other ways not available to assets. Before this event can occur all assets created within the department have to be removed. \\

% \textbf{Location added:}\\
% \textit{Location added} is triggered when a new location within the organisation is added. A location can be anything from a building, room or even a shelf. \\

% \textbf{Location removed:}\\
% \textit{Location removed} is the event occurring when a locations is removed. \\