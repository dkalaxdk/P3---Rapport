\subsection{Events}\label{ssc:events}
The interactions between the classes, defined in \autoref{ssc:classes}, can be described as events. These events have been defined and described in order to achieve a better understanding of the relation between the classes in the system.
\\\\
\textbf{Asset acquired} is the event occurring, when the zoo or a department acquires a new asset. The event involves the acquired asset and the admin, whom the asset was added by.\\

\textbf{Asset disposed of} happens when an asset is either disposed of by an admin or lost by an employee. This event affects the disposed asset and the admin, whom the disposal of the asset was handled by.\\

\textbf{Asset information updated} occurs when the information of an asset has been updated. This is done by the admin.\\

\textbf{Asset loaned out} is an event occurring, whenever an admin loans out an asset to an employee. It involves the asset, the employee to whom the asset is loaned, and the admin creating the loan.\\

\textbf{Asset returned} happens whenever an admin registers the return of an asset loaned out to an employee. This event involves the same classes as \textit{asset loaned out}, the asset, the employee and the admin.\\

\textbf{Employee added} occurs whenever an employee is hired. The only participant is the employee being hired.\\

\textbf{Employee removed} is the event where an employee gets fired or quits. Involves the employee fired.\\

\textbf{Admin access gained} occurs when an employee's access is raised to admin level. This means that the employee has gained more responsibilities, such as managing the assets. The participant is the employee who becomes an admin.\\

\textbf{Admin access revoked} happens when an admin's access level is lowered to standard employee responsibilities. This involves the admin who becomes an employee.\\

\textbf{Department activated} occurs when a department is created. The event involves the admin activating the department as well as the created department. Assets cannot exist without an active department.
 
\textbf{Department deactivated} happens when a department is deactivated and thereby becomes unavailable. Before this event can occur all assets stored within the department have to be removed. The event involves the admin deactivating the department and the department to be deactivated.\\