\subsection{Event table}\label{ssc:eventtable}
To give a visual overview of the events, classes and their relations, an event table has been constructed (see \autoref{tab:events}). The connection between an asset and an employee will be depicted by a tag with the employee's name. This way an asset can be connected to one or more employees. The  \textit{Asset loaned out} and \textit{Asset returned} events have been replaced in function by this, but the events will still be used in the analysis, as this specific use of the tag has unique effects.
\par
The \textit{Tag attached to asset} event can occur for both the \textit{Asset} and the \textit{Tag} multiple times, and therefore needs an extra class. This class illustrates the connection between a tag and an asset, and has been named \textit{Asset-tag relation}. It is instantiated every time a tag is attached to an asset and ceases to exist when this attachment is removed.

\begin{table}[H]
\centering
%\resizebox{\textwidth}{!}{%
    \begin{tabular}{|l||c|c|c|c|c||p{1.8cm}|}
        \hline
        \textbf{Event} & \textbf{Asset} & \textbf{Employee} & \textbf{Admin} & \textbf{Department} & \textbf{Tag} & \hfil \textbf{Asset-tag \newline \hfil relation}\\
        \hline
        \hline
        Asset acquired & + & & * & * & & \\
        \hline
        Asset disposed of & + & & * & * & & \\
        \hline
        Asset information updated & * & & * & & & \\
        \hline
        Asset loaned out & * & * & * & & + & \\
        \hline
        Asset returned & * & * & * & & + & \\
        \hline
        Employee hired & & + & & & & \\
        \hline
        Employee fired & & + & & & & \\
        \hline
        Admin access gained & & * & + & & & \\
        \hline
        Admin access revoked & & * & + & & & \\
        \hline
        Department activated & & & * & + & & \\
        \hline
        Department deactivated & & & * & + & & \\
        \hline
        Tag created & & & * & & + & \\
        \hline
        Tag deleted & & & * & & + & \\
        \hline
        Tag attached to asset & * & & * & & * & \hfil +\\
        \hline
        Tag detached from asset & * & & * & & * & \hfil +\\
        \hline
    \end{tabular}
%}
\caption{Event table showing which classes are involved with the different events. An event can happen once (+) or several times (*) for each class.}\label{tab:events}
\end{table}
\par

% \textbf{Loan}\\
% The \textit{Loan} class has been added to the problem domain because the events \textit{Asset loaned out} and \textit{Asset returned} both have three participating classes. These events also occur for all three classes multiple times, which is not ideal. The \textit{Loan} class connects assets and employees, and stores the objects within itself. This gives a superior way of implementing the \textit{Asset loaned out} and \textit{Asset returned} events in the system. The \textit{Loan} also offers an easier way for the \textit{Admin} to administrate relations between employees and assets.

% Beskriv at Loan blev tilføjet, da der var en mange til mange relation

% \textbf{Asset acquired:}\\
% \textit{Asset acquired} involves both Asset and Admin, Asset are acquired once. Admins can register an unlimited amount of assets, but are registered in the process. \\

% \textbf{Asset disposed of:}\\
% \textit{Asset disposed of} happens once for every assets. Admins can choose to dispose and Asset at any given time, but are registered in the process. \\

% \textbf{Asset loaned out:}\\
% \textit{Asset loaned out} ... \\