\subsection{Event table}\label{ssc:eventtable}
% Vis relationen mellem klasser og events med et event table
To give a visual overview of the events, classes and their relations, an event table has been constructed (see \autoref{tab:events}).
%To create a better overview of the events and the connections between these and the classes, an event table (\autoref{tab:events}) has been constructed.
Based on this event table, a new class has been added to the system. This class is called \textit{Loan}, and represents the loan of an asset to an employee. The reason for adding this class is explained below.
% \par
% As the admin is involved in nearly every change or addition within the system, and these actions need to be saved, these relations have been excluded from the rest of the analysis. This is done to simplify the illustrations and remove cluttering.

\begin{table}[H]
\centering
%\resizebox{\textwidth}{!}{%
    \begin{tabular}{|l||c|c|c||c|}
        \hline
        \textbf{Event} & \textbf{Asset} & \textbf{Employee} & \textbf{Admin} & \textit{\textbf{Loan}} \\
        \hline
        \hline
        Asset acquired & + & & * & \\
        \hline
        Asset disposed of & + & & * & \\
        \hline
        Asset loaned out & * & * & * & \textit{+} \\
        \hline
        Asset returned & * & * & * & \textit{+} \\
        \hline
        Employee hired & & + & & \\
        \hline
        Employee fired & & + & & \\
        \hline
        Admin access gained & & * & + & \\
        \hline
        Admin access revoked & & * & + & \\
        \hline
    \end{tabular}
%}
\caption{Event table showing which classes are involved with the different events. An event can happen once (+) or several times (*) for each class.}\label{tab:events}
\end{table}
\par
\textbf{Loan}\\
The \textit{Loan} class has been added to the problem domain because the events \textit{Asset loaned out} and \textit{Asset returned} both have three participating classes. These events also occur for all three classes multiple times, which is not ideal. The \textit{Loan} class connects assets and employees, and stores the objects within itself. This gives a superior way of implementing the \textit{Asset loaned out} and \textit{Asset returned} events in the system. The \textit{Loan} also offers an easier way for the \textit{Admin} to administrate relations between employees and assets.

% Beskriv at Loan blev tilføjet, da der var en mange til mange relation

% \textbf{Asset acquired:}\\
% \textit{Asset acquired} involves both Asset and Admin, Asset are acquired once. Admins can register an unlimited amount of assets, but are registered in the process. \\

% \textbf{Asset disposed of:}\\
% \textit{Asset disposed of} happens once for every assets. Admins can choose to dispose and Asset at any given time, but are registered in the process. \\

% \textbf{Asset loaned out:}\\
% \textit{Asset loaned out} ... \\