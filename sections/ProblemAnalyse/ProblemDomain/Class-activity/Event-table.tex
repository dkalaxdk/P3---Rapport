\subsection{Event table}\label{ssc:eventtable}
% Vis relationen mellem klasser og events med et event table
To give a visual overview of the events, classes and their relations, an event table has been constructed (\autoref{tab:events}).
%To create a better overview of the events and the connections between these and the classes, an event table (\autoref{tab:events}) has been constructed.
Based on the event table a new class has been added to the system, the \textit{Loan} class. The reason for adding the \textit{Loan} class is explained beneath the event table.
% \par
% As the admin is involved in nearly every change or addition within the system, and these actions need to be saved, these relations have been excluded from the rest of the analysis. This is done to simplify the illustrations and remove cluttering.

\begin{table}[H]
\centering
%\resizebox{\textwidth}{!}{%
    \begin{tabular}{|l||c|c|c||c|}
        \hline
        \textbf{Event} & \textbf{Asset} & \textbf{Employee} & \textbf{Admin} & \textbf{Loan} \\
        \hline
        \hline
        Asset acquired & + & & * & \\
        \hline
        Asset disposed of & + & & * & \\
        \hline
        Asset loaned out & * & * & * & + \\
        \hline
        Asset returned & * & * & * & + \\
        \hline
        Employee hired & & + & & \\
        \hline
        Employee fired & & + & & \\
        \hline
        Admin access gained & & * & + & \\
        \hline
        Admin access revoked & & * & + & \\
        \hline
    \end{tabular}
%}
\caption{Event table showing which classes are involved with the different events. An event can happen once (+) or several (*) times for each class.}\label{tab:events}
\end{table}

Now that the events and there participating classes have been illustrated in the event table, one change has been made.
\par
\textbf{Loan}\\
The \textit{Loan} has not been mentioned prior to the event table, but has been added, because the \textit{Asset loaned out} and \textit{Asset returned}, both have three participating classes. The events can both happen multiple times for all three classes, which is not ideal. To avoid this, another class has been added to the system, the \textit{Loan} class. The \textit{Loan} class connects assets and employees, and stores the id of both. The \textit{Loan} is also an easier way for the \textit{Admin} to administrate the loans.

% Beskriv at Loan blev tilføjet, da der var en mange til mange relation

% \textbf{Asset acquired:}\\
% \textit{Asset acquired} involves both Asset and Admin, Asset are acquired once. Admins can register an unlimited amount of assets, but are registered in the process. \\

% \textbf{Asset disposed of:}\\
% \textit{Asset disposed of} happens once for every assets. Admins can choose to dispose and Asset at any given time, but are registered in the process. \\

% \textbf{Asset loaned out:}\\
% \textit{Asset loaned out} ... \\