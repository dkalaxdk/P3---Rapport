\subsection{Event table}\label{ssc:eventtable}
To give a visual overview of the events, classes and their relations, an event table has been constructed (see \autoref{tab:events}). The \textbf{Loan} class has been added based on the event table, but is include within it, to illustrate its connections to the other classes.
\par

%The connection between an asset and an employee will be depicted as a tag with the employee's name. This way an asset can be connected to one or more employees. The \textit{Asset loaned out} and \textit{Asset returned} events have been replaced functionally by this, but the events will still be used in the analysis, as this specific use of the tag has unique effects.

% The \textit{Tag attached to asset} event can occur for both the \textit{Asset} and the \textit{Tag} multiple times, and therefore needs an extra class. This class illustrates the connection between a tag and an asset, and has been named \textit{Asset-tag relation}. It is instantiated every time a tag is attached to an asset and ceases to exist when this attachment is removed.

\begin{table}[H]
\centering
%\resizebox{\textwidth}{!}{%
    \begin{tabular}{|l||c|c|c|c||c|}
        \hline
        \textbf{Event} & \textbf{Asset} & \textbf{Employee} & \textbf{Admin} & \textbf{Department} & \hfil\textbf{Loan \newline \hfil}\\
        \hline
        \hline
        Asset acquired & + & & * & * & \\
        \hline
        Asset disposed of & + & & * & * & \\
        \hline
        Asset information updated & * & & * & & \\
        \hline
        Asset loaned out & * & * & * & & + \\
        \hline
        Asset returned & * & * & * & & + \\
        \hline
        Employee added & & + & & & \\
        \hline
        Employee removed & & + & & & \\
        \hline
        Admin access gained & & * & + & & \\
        \hline
        Admin access revoked & & * & + & & \\
        \hline
        Department activated & & & * & + & \\
        \hline
        Department deactivated & & & * & + & \\
        \hline
    \end{tabular}
%}
\caption{Event table showing which classes are involved in the different events. An event can happen once (+) or several times (*) for each class.}\label{tab:events}
\end{table}

As mentioned, an additional class has been added to the system, to better handle the event of an asset being loaned out to an employee.

\textbf{Loan}\\
The \textit{Loan} class has been added to the problem domain because the events \textit{Asset loaned out} and \textit{Asset returned} both have three participating classes. These events also occur for all three classes multiple times, which is not ideal. The \textit{Loan} class connects assets and employees, and stores the objects within itself. 