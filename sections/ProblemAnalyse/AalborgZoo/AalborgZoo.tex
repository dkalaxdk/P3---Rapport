\chapter{Aalborg Zoo}\label{ch:problemdefinition}
A series of semi-structured interviews have been conducted with the head of the IT department Morten Rom and technician Kasper Andersen, who make up the IT department at Aalborg Zoo. They describe a problem with keeping track of and maintaining assets within the zoo. Aalborg Zoo covers a large area and assets can be spread around the park. 
\par
Knowing the location of each individual assets comes down to the different employees in the zoo. As a result, no one has a clear picture of the assets within the zoo. Therefore Aalborg Zoo has been looking at different asset management systems. Currently the they do not have the resources to develop such a system for themselves, and has examined other options. However, none of these satisfy their requirements. They want a program that is cheap and easy to use, without presenting the user with a large number of fields to fill out for each new entry into the system. 
%Currently the zoo doesn't have the resources to develop such a system for themselves, and has examined other options. However, none of the systems the zoo has looked at, makes it easy for the user to customize and modify templates.
\par
During the interview process, the head of the IT department expressed a desire for an asset management system with the following requirements. These are presented using the MoSCoW method for prioritization:
\par





%\begin{itemize}
%    \item Make it possible for an administrator to record when employees borrow assets, or associate an asset with a location. This should be done by adding special tags, like a user or location tag.
    
%    \item Add uncategorized tags with additional relevant information to assets. This could be the status of the asset, if for example it is under maintenance. 
    
%    \item A log of the changes made to assets, along with information about who made the change and when it was made. It should also be logged when an asset is added or removed. 
    
%    \item Templates for assets, with only the necessary fields that all assets of that template need to have. For example a switch needs fields with ID, IP-address, and password. All other relevant information can be added through tags.
%\end{itemize}

%The reason for the small number of required fields, is due to the clients wishes. They expressed a dislike for other asset management systems that required too many fields to be filled for each asset. They also wanted as few clicks of the mouse as possible.