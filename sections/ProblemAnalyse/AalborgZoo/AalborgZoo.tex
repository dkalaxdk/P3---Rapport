\section{Aalborg Zoo}\label{ch:problemdefinition}
Focusing on a specific department at a specific company, has limited the magnitude of the project. The collaborator of this project has been the IT department at Aalborg Zoo, who asked for an Asset Management System that is more dynamic than the existing solutions. Therefore, the needs and wants of Aalborg Zoo has been a high priority, as they have been seen as the customer of the finished product.
\par
The collaboration has been based on a series of semi-structured interviews, which have been conducted with the head of the IT department, Morten Rom, and technician Kasper Andersen who make up the IT department at Aalborg Zoo, and have been representatives of the IT department and partly the zoo as a whole. They have described a problem with keeping track of and maintaining assets within the zoo due to the physical size of the zoo and the number of different assets, departments, and employees. Their system for managing their assets has been to type in information that was thought relevant at the point of acquisition. This system is flawed, as the pieces of information added by different employees change from time to time, and information about the assets is not being kept up to date. Therefore, a proper Asset Management System could be a good addition to the company.
\par
During the interviews, the representatives where asked, if they had looked for alternatives, as systems solving the problem already exists. To this they answered that the existing solutions provide too many fields to be filled out with information. Here a field could be a textbox, checkbox, date field, etc., meant to contain information about the asset. They instead wanted a dynamic system with only the information they needed for every specific asset, resulting in a cleaner, more manageable interface for adding a new asset.
\par
They have also expressed an interest in logging changes and the ability to export a list of assets as well as the log. 

\todo[inline]{Overgang til problemformuleringen}

% Knowing the location of each individual assets comes down to the different employees in the zoo. As a result, no one has a clear picture of the assets within the zoo. Therefore Aalborg Zoo has been looking at different asset management systems. Currently they do not have the resources to develop such a system for themselves, and has examined other options. However, none of these satisfy their requirements. They want a program that is cheap and easy to use, without presenting the user with a large number of fields to fill out for each new entry into the system. 

%\begin{itemize}
%    \item Make it possible for an administrator to record when employees borrow assets, or associate an asset with a location. This should be done by adding special tags, like a user or location tag.
    
%    \item Add uncategorized tags with additional relevant information to assets. This could be the status of the asset, if for example it is under maintenance. 
    
%    \item A log of the changes made to assets, along with information about who made the change and when it was made. It should also be logged when an asset is added or removed. 
    
%    \item Templates for assets, with only the necessary fields that all assets of that template need to have. For example a switch needs fields with ID, IP-address, and password. All other relevant information can be added through tags.
%\end{itemize}

%The reason for the small number of required fields, is due to the clients wishes. They expressed a dislike for other asset management systems that required too many fields to be filled for each asset. They also wanted as few clicks of the mouse as possible.