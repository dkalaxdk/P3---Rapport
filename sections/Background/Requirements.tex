\section{Requirements}\label{sc:requirements}
Based on the interviews with the zoo, a list of requirements has been formulated. According to the MoSCoW method \citep[chap 7.1]{DEB}, these requirements have been divided into categories based on their priority. The final requirement specification can be seen in \autoref{tab:moscow}. This prioritisation has been approved by the client and will dictate the development of the system.

\begin{longtable}{p{3.2cm} p{10cm}}
    \renewcommand{\arraystretch}{2.0}
        \\
        \hline
        \textbf{Must have} & 
        \vspace*{-7mm}
        \begin{enumerate} \itemsep0em
        
            \item Assets can be added, removed, and edited. Fields can be dynamically added to these.
            
            \item Assets can be searched through, using either their id, identifier or name.
            
            \item Information on users can be imported from the Active Directory, through a CSV file.
            
        \end{enumerate}
        \\
        \hline
        
        \textbf{Should have} & 
        \vspace*{-7mm}
        \begin{enumerate} \setcounter{enumi}{3} \itemsep0em 
            
            \item Tags can be added, removed, and edited. They can also be attached to assets, and have fields dynamically added to them.
            
            \item Assets can be searched for by their attached tags.
            
            \item Assets and log entries can be exported to a file, potentially based on a search.
            
            \item Admins and employees can add comments to assets. Admins can remove and edit all comments, while employees can only edit and remove their own comments.
            
            \item Comments for each asset can be accessible from within the specific asset.
            
            \item All changes to assets and tags will be logged.
            
            \item Fields can be set to a certain type and can be marked as required. It will be possible for a field to have a default value.
            
            \item The design will be minimalistic.
            
            \item Tags can be presented in different colors. Child tags will inherit the color of their parent tag. Colors will be assigned randomly if not chosen by the user.
            
        \end{enumerate}
        \\
        \hline
        
        \textbf{Nice to have} &     
        \vspace*{-7mm}
        \begin{enumerate} \setcounter{enumi}{12} \itemsep0em 
        
            \item Departments can be added, removed, and edited. Each user can set their default department.
            
            \item Tags are department specific.
            
            \item It will be possible to select multiple assets. The selected assets can then be exported to a file.
            
            \item When searching for assets, a list of tags on the found assets will be shown. From this list it will be possible to hide assets by deselecting tags.
            
            \item Assets can have an expiration date, and the system will inform the user of this.
            
            \item Tags can contain functionality.  
            
            \item Employees can request an asset. The admins can view, approve, or deny all requests for assets.
            
        \end{enumerate}
        \\
        \hline
        \textbf{Won't have} & 
        \vspace*{-7mm}
        \begin{enumerate} \setcounter{enumi}{19} \itemsep0em 
        
            \item AND, OR, and parentheses can be used to enable complex searching.
            
            \item System functionality can be linked to barcodes.
            
        \end{enumerate}
        \\
        \hline
    \caption{MoSCoW table over requirements}
    \label{tab:moscow}
   
\end{longtable}

Based on these requirements a system definition has been formulated, to define the overall system.