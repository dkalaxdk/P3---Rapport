\section{Aalborg Zoo}\label{sc:aalborgZoo}
The collaboration with Aalborg Zoo has been based on a series of semi-structured interviews. These have been conducted with the head of the IT department, Morten Rom, and technician Kasper Andersen who make up the IT department at Aalborg Zoo. They have described a problem with keeping track of and maintaining assets within the zoo, due to the physical size of the zoo and the number of different assets, departments, and employees.
\par
Their current system for managing their assets has been to add the relevant information about the assets to an Excel document at the time of acquisition. For the IT department this system is flawed, as the information added by different employees can vary from asset to asset, and this information is difficult to maintain.
\par
During the interviews, the representatives were asked, if they had looked for alternatives, as systems solving their problem already exist. To this they answered that the existing solutions provide too many required and predefined fields. Here a field could be a text box, checkbox, date field, etc., meant to contain information about the asset. Instead they wanted an adaptable system with only the information they needed for every specific asset, resulting in a cleaner, more manageable interface. 
\par
To ensure adaptability in the system, the fields should be able to be added and removed as they are needed.
\par
During the interviews the representatives' ideas for the system were brought forward. These ideas included functionalities such as tagging, expiring assets and a log. \\
Tagging was one of the first ideas which were mentioned by the representatives, as this would be an easy way to add descriptors and fields to assets, while still keeping the interface manageable and functional.
\par
Another functionality mentioned was the ability to add expiration dates to assets. This could be a way for Aalborg Zoo to ensure that specific assets were changed or maintained after a preset period of time.
\par
The representatives mentioned a log containing information on the changes made to assets and their properties, as they would like to know which of the system's users modified specific assets. This would allow them to find possible mistakes and fixing these. 
\par
To further expand on tagging, the idea came forth to add functionality to certain tags. This would allow the tags to perform actions, such as notifying the IT department about changes in the system. \\
To make it easier to keep an overview of the tags added to the system, the idea of grouping tags into categories was mentioned.
\par
For the logging to be functional, there was an obvious need for a way to identify the individual users using the system. To solve this, and to restrict certain functionalities within the system, the head of the IT department mentioned that they could export a list of the employees in the zoo as a Comma Separated Values (CSV) file, so the application could use this to identify users. As a result the application would need a way of reading the exported file, and adding the employees to the system. 
\par
There was also a desire for the employees of the zoo to be able to leave comments on assets, hereby giving the IT department updates with information about an asset in the system.
\par
The ability to export a report of the assets was also mentioned, as a way of having documentation of the registered assets, outside the system. The export should be in a CSV file, as this can be read by multiple different programs.
\newpage