\section{System Definition}\label{sc:SystemDefinition}
The primary functionality of the system is to give the user the ability to add, edit, remove, and search through all assets. The search will be based on the assets' unique IDs, names, or identifiers specified by the user, such as tags and fields, and is limited to the selected department. When adding a new asset to the system, the user will be able to add predefined tags to the asset. These tags can contain fields that should be filled out for the specific assets.
\par
The system will have the functionality to tag assets with attributes such as their location and the person borrowing the asset. The user will be able to add new tags to the system and assign these to parent tags. The parent tags will act as tag categories, which the user can define. 
\par
Another functionality of the system is logging changes. The system will be able to provide a history of events for each asset. These include adding an asset, editing an asset, and attaching and detaching tags to an asset. The user will also be able to add comments to assets, describing events such as updating the operating system of a computer or installing new software to an asset. Every event and comment will have the username of the person who made the change attached.
\par
The system is developed to be primarily used by the IT department, and has a group of users who can manage the system, logging in automatically using their Active Directory (AD) identity. Other departments will also have access to the system. All employees will be able to view and comment on the assets, but only the assigned administrators will have the ability to manage them. 
\par
The interface will be simple and easy to use, changing the search results based on the selected department. The user will be able to generate a list of assets and export it to a file.
\par
The system will be programmed using the C\# language and will be developed for a Windows PC. The system will be communicating with a database and have a graphical user interface.
\par
The primary objects in the problem domain are:
\begin{itemize}
    \item \textbf{Asset}: Corporate assets in Aalborg Zoo, such as computers, switches, phones, etc.
    
    \item \textbf{Employee}: Every employee can access the system and see every asset. Some employees can be granted further access to functionality, at which point they become an admin.
    
    \item \textbf{Admin}: Users with authority to manage and lend assets to other employees.
    
    \item \textbf{Department}: Different departments within the zoo, which oversee multiple assets.
\end{itemize}
While tags are an essential part of the system, they are not objects in the problem domain and will not be discussed again till \autoref{sc:model_component}.
\par
A more concise version of the system definition can be found in the appendix as a FACTOR criterion (see \autoref{app:factor}).