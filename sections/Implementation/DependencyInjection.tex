\section{Dependency Injection} \label{sc:DependencyInjection}
Dependency Injection is utilized to ensure a loose coupling between the classes. This has been done by making classes take interfaces in their constructor parameters, instead of concrete classes. On top of this, any dependencies that needs to be used across the system, can be accessed through the \textit{Features} class (see \autoref{sc:features}). If the implementation of an interface needs to be changed, it only needs to be changed in the \textit{Features} class or in one of its components.
\par
To help explain Dependency Injection, an example from the program will be used, which can be seen in \autoref{code:TagEditorViewModel}.

\begin{listing}[H]
\begin{minted}[frame=lines, framesep=3mm, baselinestretch=1, linenos, bgcolor=LightGray, escapeinside='', breaklines]{csharp}

private ITagController _tagController;

public TagEditorViewModel(ITagController tagController)
{
    _tagController = tagController;
    
    ...
}

\end{minted}
\captionof{listing}{TagEditorViewModel Example.}
\label{code:TagEditorViewModel}
\end{listing}

As seen in \autoref{code:TagEditorViewModel}, the class \textit{TagEditorViewModel} requires the instance of a class that implements the interface \textit{ITagController}. In this case \textit{ITagController} is implemented by the class \textit{TagController} which could be instantiated within \textit{TagEditorViewModel}, but is instead instantiated outside the class and given as a parameter. As a result of doing this, the specific implementation of \textit{ITagController} is irrelevant, as long as it implements the interface. 
\par
Depending on abstractions instead of concretions is the Dependency Inversion Principle, one of the five SOLID principles of Object Oriented Design \citep{AgilePPP}. The result of this is a loose coupling between \textit{TagEditerViewModel} and \textit{ITagController}, as none of the classes rely on the specific implementation of the other class.
\par
If the instantiation of a concrete class implementing the interface happened within \textit{TagEditorViewModel}, then the class would need to know the parameters of the class implementing \textit{ITagController}, which would result in a higher coupling.
\par

Loose coupling makes it easier to test the code via Unit Testing, as the interfaces enables the use of mocks, stubs, and fakes. These are ways of faking data that is not relevant for the current test. This makes the tests smaller and easier to verify, as they do not necessarily require interactions with the underlying data structures or database structures. Instead it allows for faking the implementation of an interface, and define what the members should return, making the tests independent of any underlying infrastructure.
\par
It is also possible to simply check whether certain functions have been called as expected. A description of how this is used in the project, is explained in \autoref{sc:UnitTest}.
\par
Another benefit of Dependency Injection is that it becomes very simple to change the implementation of a dependency. This will be possible if the code is designed in accordance with the Liskov Substitution Principle, which states that "\textit{Subtypes must be substitutable for their base types}" \citep{AgilePPP}. 

\begin{listing}[H]
\begin{minted}[frame=lines, framesep=3mm, baselinestretch=1, linenos, bgcolor=LightGray, escapeinside='', breaklines]{csharp}

protected FieldListController(FieldContainer element)
{
    ...
}

\end{minted}
\captionof{listing}{An example of a class made after the Liskov Substitution principle.}
\label{code:FieldControllerLiskov}
\end{listing}

\autoref{code:FieldControllerLiskov} is an example of the use of the Liskov Substitution Principle. Both of the classes \textit{Asset} and \textit{Tag} inherit from the class \textit{FieldContainer}. Either one of these classes can be given as a parameter to \textit{FieldsListController}. 
\par
This principle can also be extended to interfaces, meaning that it must be possible to change one implementation of an interface to another, without causing the code to break \citep{seemann2019dependency}. 
\par
When object composition happens in the composition root, which in this case is \textit{Features}, and all classes depend on abstractions, it is only necessary to change the implementation of an interface or base class in one place.

%https://www.tutorialsteacher.com/ioc/dependency-injection

%Makes it possible to have loose coupling\\
%Loose coupling in turn makes the code easier to test via the use of mocks/stubs/fakes\\
%Makes it easy to make code that follows the principles of SOLID\\
%Enables Open/closed by making Decorator pattern easy (bruger vi ikke)\\
%Decorator also makes single responsibility easier\\
%By depending on abstraction makes liskov substitution principle (Look it up) easy\\
        % http://butunclebob.com/ArticleS.UncleBob.PrinciplesOfOod
%Dependency inversion is the whole point\\
%Dependency inversion makes it easier/possible to to TDD\\
%Having object composition and lifetime elsewhere can increase maintainability\\
        % kilde: Dependency Injection Principles, practices, Patterns (Steven van Deursen & Mark seemann)
