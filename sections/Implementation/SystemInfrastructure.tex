\section{System Infrastructure} \label{sc:system_infrastructure}
% The starting point of the application can be found in \textit{App.xaml.cs}. Here the class \textit{Features} creates the main window, which in turn creates the main view model. \textit{MainViewModel} and \textit{Features} are responsible for everything going on behind the scenes, such as navigating and changing department. 
% \par
% Both of these classes have too much responsibility, but were designed this way in order to keep the program simple and understandable.

\subsection{MainViewModel} \label{ssc:main_view_model}
% System root / entry point of the program

% View model for the window itself, showing menus and other stuff

% Provides a Frame where all content is presented

% Exposes commands used to navigate between the primary pages.



\subsection{Features Class} \label{ssc:features}
% Main responsibility is Repositories

% Simplifies the program, by being the single source to MainViewModel and its public properties/methods

% Navigating to and from pages using History 

% Creating pages (and one window), providing the repositories and creating the controllers (And helpers)

% Miscellaneous stuff: Notifications, prompts, current session and department. The things that Main is used most for.

% Problems: Too much responsibility, should be split up. => Singleton (Tests, good practice, dependency injection)
