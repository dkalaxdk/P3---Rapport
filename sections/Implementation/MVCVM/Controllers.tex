\subsection{Controllers} \label{sc:Controllers}
Controllers contain the functionality related to updating and removing data. The controllers add functionality to manipulate the models, as a model does not contain any functionality in itself. 
\par
An example of an interface implemented by a controller can be seen in \autoref{code:IAssetController}. The \textit{IAssetController} contains a public \textit{Asset} property, which is the asset that the controller manipulates. It also contains definitions of the functions related to saving, updating, removing, attaching and detaching tags to and from the asset, as well as information related to displaying the asset to the view model.

\begin{listing}[H]
\begin{minted}[frame=lines, framesep=3mm, baselinestretch=1, linenos, bgcolor=LightGray, escapeinside='', breaklines]{csharp}

public interface IAssetController : IFieldListController
{
    Asset ControlledAsset { get; set; }
    List<ITagable> CurrentlyAddedTags { get; set; }

    void AttachTags(List<ITagable> tags);
    void AttachTags(ITagable tag);
    void DetachTags(List<ITagable> tags);

    bool Save();
    bool Update();
    bool Remove();

    void UpdateFieldContent();

    void RevertChanges();
}

\end{minted}
\captionof{listing}{IAssetController example.}
\label{code:IAssetController}
\end{listing}

The IAssetController interface implements the IFieldsListController, as an \textit{Asset} has fields. Therefore an implementation of IAssetController also needs to be able to control fields.
\par
The members of the \textit{IAssetController} interface are implemented by the \textit{AssetController} class partially shown in \autoref{code:AssetController}. The class includes the connection to the database, as well as methods for manipulating the model.
\par
The goal of the controller is to take the data input from the view model, manipulate the model, and send it to the appropriate repository, as well as fetching the data from the repository and presenting it to the view model (see \autoref{sc:RepositoryPatern}).

\begin{listing}[H]
\begin{minted}[frame=lines, framesep=3mm, baselinestretch=1, linenos, bgcolor=LightGray, escapeinside='', breaklines]{csharp}

public class AssetController : FieldListController, IAssetController
{
    public AssetController(Asset asset, IAssetRepository assetRepository, Session session) : base(asset)
    {
        ControlledAsset = asset;

        _assetRepository = assetRepository;
        _session = session;

        ControlledAsset.DeSerializeFields();
        LoadTags();
    }
    ...
}
\end{minted}
\captionof{listing}{AssetController code example.}
\label{code:AssetController}
\end{listing}

To understand the specific implementation of a controller, the constructor of the \textit{AssetController} class (see \autoref{code:AssetController}) is examined. The controller gets the \textit{Asset} it needs to manipulate and makes it accessible to the view model.
\par
As seen in \autoref{code:AssetController} the controller inherits from \textit{FieldListController}, the interface of which is seen in \autoref{code:IFieldListController}. This abstract class contains the functionality related to controlling fields.

\begin{listing}[H]
\begin{minted}[frame=lines, framesep=3mm, baselinestretch=1, linenos, bgcolor=LightGray, escapeinside='', breaklines]{csharp}
public interface IFieldListController
{
    bool SerializeFields();

    bool AddField(Field field);
    bool RemoveField(Field inputField);
    
    bool RemoveTagRelationsOnFields(ITagable tag);
}
\end{minted}
\captionof{listing}{IFieldListController example.}
\label{code:IFieldListController}
\end{listing}

The \textit{AssetController} makes the functionality of adding, removing, and editing an \textit{Asset} available to the \textit{AssetEditorViewModel}. This allows for the view model to only use the methods defined by the \textit{IAssetController} interface, which allows for separation of concerns and functionality. An example of the implementation of a view model can be seen in the following subsection.