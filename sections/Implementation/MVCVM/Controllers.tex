\subsection{Controllers} \label{sc:Controllers}
As mentioned in \autoref{sc:MVCVM} controllers contains the functionality related to modifying, updating or removing data. The controllers add functionality to the models, as a model does not contain any functionality in itself. \par
An example of an interface to a controller can be seen in \autoref{code:IAssetController}. The \textit{IAssetController} contains a public \textit{Asset} property, which is the asset that the controller manipulates. \par
The \textit{IAssetController} interface also contains definitions of the functions related to saving, updating, removing, tagging and untagging tags from the asset, as well as information related to displaying the asset to the view model.

\begin{listing}[H]
\begin{minted}[frame=lines, framesep=3mm, baselinestretch=1, linenos, bgcolor=LightGray, escapeinside='', breaklines]{csharp}

public interface IAssetController : IFieldListController
{
    Asset ControlledAsset { get; set; }
    List<ITagable> CurrentlyAddedTags { get; set; }

    string Name { get; set; }
    string Identifier { get; set; }
    string Description { get; set; }

    bool AttachTag(ITagable tag);

    bool DetachTag(ITagable tag);

    bool Save();
    bool Update();

    bool Remove();
}

\end{minted}
\captionof{listing}{IAssetController example. The IAssetController interface implements the IFieldsListController, as an \textit{Asset} has fields. Therefore an implementation of IAssetController also need to be able to control fields.}
\label{code:IAssetController}
\end{listing}
\par
The members of the \textit{IAssetController} interface are implemented by the \textit{AssetController} class partially shown in \autoref{code:AssetController}, this class contains the information related to utilizing the members of the interface, this include the connection to the database, as well as any other controllers that may be needed to execute the functionality. \par
The goal of the controller is to take the data input from the view model, connect it to the model and send it to the appropriate repository (\autoref{sc:RepositoryPatern}). \par

\begin{listing}[H]
\begin{minted}[frame=lines, framesep=3mm, baselinestretch=1, linenos, bgcolor=LightGray, escapeinside='', breaklines]{csharp}
public AssetController(Asset asset, IAssetRepository assetRepository) : base(asset ?? new Asset())
{
    if (asset == null)
    {
        ControlledAsset = new Asset();
    }
    else
    {
        ControlledAsset = asset;
    }
    ControlledAsset.DeSerializeFields();

    Name = ControlledAsset.Name;
    Identifier = ControlledAsset.Identifier;
    Description = ControlledAsset.Description;
    
    NonHiddenFieldList = ControlledAsset.FieldList.Where(f => f.IsHidden == false).ToList();
    HiddenFieldList = ControlledAsset.FieldList.Where(f => f.IsHidden == true).ToList();
    
    _assetRepository = assetRepository;
    CurrentlyAddedTags = _assetRepository.GetTags(ControlledAsset).ToList();
    LoadTags();
}

\end{minted}
\captionof{listing}{AssetController example.}
\label{code:AssetController}
\end{listing}

To understand the specific implementation of a controller, the constructor of the \textit{AssetController} class(\autoref{code:AssetController}) is examined. The controller gets the \textit{Asset} it needs to manipulate and connects its values to public properties of the controller class, in order to make them available to a view model. \par
As seen on \autoref{code:AssetController} the controller inherits from a base, the interface to the base, can be seen in \autoref{code:IFieldListController}, the \textit{FieldListController} contains the functionality related to controlling fields, the \textit{AssetController} inherits this functionality.

\begin{listing}[H]
\begin{minted}[frame=lines, framesep=3mm, baselinestretch=1, linenos, bgcolor=LightGray, escapeinside='', breaklines]{csharp}
public interface IFieldListController
{
    List<Field> NonHiddenFieldList { get; set; }
    List<Field> HiddenFieldList { get; set; }

    bool SerializeFields();

    bool AddField(Field field,FieldContainer fieldContainer = null);

    bool RemoveField(Field inputField, FieldContainer fieldContainer = null);

    bool RemoveFieldRelations(ulong TagId);
}
\end{minted}
\captionof{listing}{IFieldListController example.}
\label{code:IFieldListController}
\end{listing}

The \textit{AssetController} makes the functionality of adding/removing/editing an \textit{Asset} available to the \textit{AssetEditorViewModel} this allows for the view model to only use the methods defined by \textit{IAssetController} interface, and therefore it does not need to know the specific implementation of the \textit{Asset} model. This in return allows for separation of concerns, and functionality. A example of the implementation of a view model can be seen in the following chapter.
