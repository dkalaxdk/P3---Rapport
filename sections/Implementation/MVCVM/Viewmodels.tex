\subsection{View models} \label{sc:ViewModels}
The primary purpose of the view models is to expose the data provided by the controllers to the views. This data is then presented to the user by the view. The view model is also responsible for changing the view and moving from one page to another. Any actions performed by the user through the view is also handled in the view model. 
\par

View models do not implement any interfaces. This is because there is one view model per view, making them tightly coupled, which eliminates the need for a common interface.
\par
To illustrate how the functionalities of a view model are implemented, the \textit{AssetEditorViewModel} class will be used as an example (see \autoref{code:AssetEditorViewModel}). This view model is used when a user creates or edits an \textit{Asset}. It therefore has functionalities that update both the view and calls the respective controller.
\par
The view needs certain information to be available through public properties in the view model. When these properties are changed, the view model can call the method \textit{OnPropertyChanged} to update the view according to the changes. This method is inherited from the \textit{BaseViewModel} class, which all view models inherit from.

\begin{listing}[H]
\begin{minted}[frame=lines, framesep=3mm, baselinestretch=1, linenos, bgcolor=LightGray, escapeinside='', breaklines]{csharp}

public class AssetEditorViewModel : BaseViewModel
{
    ...
    public ObservableCollection<ITagable> AppliedTags { get; set; } = new ObservableCollection<ITagable>();
    ...

    public string Name
    {
        get => _assetController.ControlledAsset.Name;
        set => _assetController.ControlledAsset.Name = value;
    }
    ...
    private IAssetController _assetController { get; set; }
    ...
    public ICommand SaveCommand { get; set; }
    ...

    public AssetEditorViewModel(IAssetController assetController, TagHelper tagHelper)
    {
        ...
        // Commands
        SaveCommand = new RelayCommand(() => SaveAndExit());
        ...
    }
    ...
}

\end{minted}
\captionof{listing}{AssetEditorViewModel example.}
\label{code:AssetEditorViewModel}
\end{listing}

When the user performs an action in the view, the view model needs to execute the command associated with this action. The command then calls a method to implement the changes that the user wants to make through the controller. 
\par
Commands in WPF are classes that implement the \textit{ICommand} interface. An example of a command is the \textit{RemoveDepartmentCommand} shown in \autoref{code:RemoveDepartmentCommandExample}.

\begin{listing}[H]
\begin{minted}[frame=lines, framesep=3mm, baselinestretch=1, linenos, bgcolor=LightGray, escapeinside='', breaklines]{csharp}

class RemoveDepartmentCommand : ICommand
{
    private MainViewModel _main;
    private Department _department;
    private readonly Func<bool> _func;

    public event EventHandler CanExecuteChanged;

    public RemoveDepartmentCommand(MainViewModel main, Func<bool> func)
    {
        _main = main;
        _func = func;
    }

    public bool CanExecute(object parameter)
    {
        if (_func != null)
            return _func();

        return true;
    }

    public void Execute(object parameter)
    {
        ...
    }
    ...
}

\end{minted}
\captionof{listing}{The RemoveDepartmentCommand is an example of a WPF command as it implements the \textit{ICommand} interface.}
\label{code:RemoveDepartmentCommandExample}
\end{listing}

However it is not necessary to define a new class for every possible action the user can take. A \textit{RelayCommand} is used throughout the code as it makes it possible to call an existing method in its own execute method. This is done by instantiating a \textit{RelayCommand} with an action as a parameter.

\begin{listing}[H]
\begin{minted}[frame=lines, framesep=3mm, baselinestretch=1, linenos, bgcolor=LightGray, escapeinside='', breaklines]{csharp}

SearchCommand = new RelayCommand(() => Search());

\end{minted}
\captionof{listing}{Example of RelayCommand.}
\label{code:RelayCommandExample}
\end{listing}

\autoref{code:RelayCommandExample} shows assigning the \textit{ICommand} to the \textit{SearchCommand}, by creating a new \textit{RelayCommand} with the method \textit{Search} given as a parameter. The parameter is given in the form of a lambda expression, as the \textit{RelayCommand} constructor needs a parameter of the type \textit{Action} \citep{Action}. 
\par
An example of a view model is shown in \autoref{code:AssetEditorViewModel}. Most of the code is not shown, as it is not relevant in the context of connecting the view and model. It shows how the view model exposes properties like \textit{AppliedTags} and \textit{Name} for a view to use. 