\subsection{View models} \label{sc:ViewModels}
As mentioned in \autoref{sc:MVCVM} the primary purpose of the view models is to present the data provided by the controllers and repositories. This data is then used by the view to present to the user. The view model is also responsible for changing the view and moving from one page to another. Any commands issued by the user through the view is also handled in the view model. 
\par 
View models do not implement any interfaces. This is because there is one view model per view. This means that it is never necessary to replace the view model, as long as the same functionality is needed. Which eliminates the need for a common interface.
\par
To illustrate how the functionalities of a view model are implemented the \textit{AssetEditorViewModel} class will be used as an example. This view model is used when a user creates or edits an \textit{Asset}. It therefore has functionalities that update both the view and calls the respective controller. 
\par
The view needs certain information to be available, through public properties in the view model. When these properties are changed, the view model can call the method \textit{OnPropertyChanged} to update the view according to the changes. 
\par
When the user executes an action in the view, the view model needs to execute the command associated with this action. A command is connected to an action that the user can take within the view, it then calls a method to implement the changes that the user want to make through the controller. \par
Commands in WPF are classes that implement the \textit{ICommand} interface. 
An example of a command is the \textit{RemoveDepartmentCommand} shown in \autoref{code:RemoveDepartmentCommandExample}.

\begin{listing}[H]
\begin{minted}[frame=lines, framesep=3mm, baselinestretch=1, linenos, bgcolor=LightGray, escapeinside='', breaklines]{csharp}

 class RemoveDepartmentCommand : ICommand
    {
        private MainViewModel _main;
        private Department _department;
        private readonly Func<bool> _func;

        public event EventHandler CanExecuteChanged;

        public RemoveDepartmentCommand(MainViewModel main, Func<bool> func)
        {
            _main = main;
            _func = func;
        }

        public bool CanExecute(object parameter)
        {
            if (_func != null)
                return _func();

            return true;
        }

        public void Execute(object parameter)
        {
            ...
        }
        ...
    }

\end{minted}
\captionof{listing}{The RemoveDepartmentCommand is an example of a WPF command as it imnplements the \textit{ICommand} interface.}
\label{code:RemoveDepartmentCommandExample}
\end{listing}

However it is not necessary to define a new class for every possible action the user can take. By creating a \textit{RelayCommand}, it is possible to execute a method instead. This is done by creating the \textit{RelayCommand} with a method as a parameter. The given method can not have a return value.
\\

\begin{listing}[H]
\begin{minted}[frame=lines, framesep=3mm, baselinestretch=1, linenos, bgcolor=LightGray, escapeinside='', breaklines]{csharp}

SearchCommand = new RelayCommand(() => Search());

\end{minted}
\captionof{listing}{Example of RelayCommand.}
\label{code:RelayCommandExample}
\end{listing}

\autoref{code:RelayCommandExample} shows assigning the \textit{ICommand} to the \textit{SearchCommand}, by creating a new \textit{RelayCommand} with the method \textit{Search} given as a parameter. The parameter is given in the form of a lambda expression, as the \textit{RelayCommand} constructor needs a parameter of the type \textit{Action}. \par

An example of a view model is shown in \autoref{code:TagEditorViewModel}. Most of the code is not shown, as it is not relevant. It shows exposing properties like \textit{AppliedTags} and \textit{Name} that a view can use. The example also clarifies how the view model receives a controller through the constructor, and how it assigns a method to a command.

\begin{listing}[H]
\begin{minted}[frame=lines, framesep=3mm, baselinestretch=1, linenos, bgcolor=LightGray, escapeinside='', breaklines]{csharp}

public class AssetEditorViewModel : BaseViewModel
{
    ...
    public ObservableCollection<ITagable> AppliedTags { get; set; } = new ObservableCollection<ITagable>();
    ...

    public string Name
    {
        get => _assetController.Name;
        set => _assetController.Name = value;
    }
    ...
    private IAssetController _assetController { get; set; }
    ...
    public ICommand AddTagCommand { get; set; }
    ...

    public AssetEditorViewModel(IAssetController assetController, TagHelper tagHelper)
    {
        ...
        // Commands
        ...
        AddTagCommand = new RelayCommand(() => TagSearch());
        ...
    }
    ...
}

\end{minted}
\captionof{listing}{AssetController example.}
\label{code:AssetEditorViewModel}
\end{listing}