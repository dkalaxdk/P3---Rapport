\chapter{Future work}
Although the system has been judged to solve the problems of the client, there is still room for improvement. In this chapter a few ideas for such improvements have been described. 

\section{Optimization}
As mentioned in the conclusion, the developed application meets most of the requirements (see \autoref{ch:conclusion}). Therefore the next obvious part of future work, would be to finish the requirements mentioned  in \autoref{sc:requirements}, as well as optimizing the core parts of the code, such as minimizing the calls to the database. 
\par
Another way to optimize the source code is refactoring, with the purpose of making it more maintainable, as well as making it easier to add new functionalities. Upon refactoring it would also be beneficial to fully implement some of the design principles that were planned from the start, such as Dependency Injection (see \autoref{sc:DependencyInjection}) and the SOLID principles \citep{AgilePPP}.
\par
It would be beneficial to optimize the application, for instance by removing unnecessary calls to procedures. This would allow for further expansion, without compromising performance.

\section{Expanding the software}
Upon expanding the application certain functionalities would be beneficial to implement, such as upgrading the database structure, and creating web and mobile interfaces.
\par
The web interface would allow for the application to run on a wider range of devices, but would also require the application to be re-written to support a client server structure. 
\\\\
With these expansions it is possible to reach a larger user base. Because the application is a general asset management system, coded to be dynamic and costumizable, it could be useful to clients other than Aalborg Zoo.