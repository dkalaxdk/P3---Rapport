\section{Connecting Components} \label{sc:connecting_components}
The component have been defined in the above sections and their connections and desired interactions have been described and depicted in the following section.
\par
The model and function components have a few dependencies between them. These dependencies have been converted to connections in a way that gives the most cohesion in the system and the lowest coupling. On top of these criteria, an intuitive design and connection between the components has been striven towards.
\par
The connection between the \textit{Asset}, \textit{Tag}, and \textit{Department} classes and the \textit{MaintainModelData} strategy class is an aggregation. This is because the concrete strategies handles objects of their designated type as a collection, which is known to the concrete strategy.
\par
The \textit{Admin} class is connected with a call relation to the \textit{AssetController} class, as the \textit{Admin} just calls the function within the \textit{AssetController}.
\par
The \textit{Employee} class has a connection to the \textit{Session} class in the function component to ensure that the an employee can access the function belonging to the session. The connection exists as a call, as the relation between the two classes' only purpose is giving the \textit{Employee} access to the functions.\\

The above descriptions has described the connections between the model and function components. The decisions reasoned above have ended in the following component diagram.