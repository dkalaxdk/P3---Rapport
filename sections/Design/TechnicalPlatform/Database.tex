\subsection{Database}\label{ssc:tech_database}
As the system needs to be used on multiple devices at the same time, a central storage control needed to be implemented. To ensure a reliable storage solution, a database has been chosen to act as the central storage system. Aalborg Zoo currently uses a Microsoft SQL 2012 server, but is an older version and does not support JSON indexation, which allows JSON elements to be searched through as virtual columns \cite{MySQLJSON}. JSON indexation allows a more versatile system, as well as making it possible to implement some of the requirements, such as \textit{Function tags} (see \autoref{sc:requirements}), at a later date.\\
Because JSON indexation was a requirement for the chosen database, MySQL was chosen as the database type for the system. MySQL has also been chosen because it is free to use and is supported by a large user-base and has comprehensive documentation.
\par
The database is configured as a relational database \cite{RelationalDB} so each element is only present once, and any relation between objects are defined with relations between ID's. This makes the database more storage efficient, but can make it slower than a non relational database, upon scaling. This can, however, be combated by smart model design and has therefore not been classified as a major issue.\\