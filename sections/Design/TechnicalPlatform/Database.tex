\subsection{Database}\label{ssc:tech_database}
As the system needs to be used on multiple devices at the same time, a central storage solution needed to be implemented. To ensure reliable storage, a database has been chosen to act as the central storage system. Aalborg Zoo currently uses a Microsoft SQL 2012 server, but it is an older version and does not support JSON indexation, which allows JSON elements to be searched through as virtual columns \citep{MySQLJSON}. 
JSON indexation allows a more versatile system, as well as making it possible to implement some of the requirements, such as \textit{Function tags} (see \autoref{sc:requirements}).
\par
Because JSON indexation was a requirement for the chosen database, MySQL 5.7 was chosen as the database type for the system. MySQL has also been chosen because it is free to use and is supported by a large user-base and has comprehensive documentation. This decision has been approved by the client, as it is not a big investment and they are able to run it on their current servers.
\par
The database is configured as a relational database so each element is only present once, and any relation between objects are defined with relations between ID's of the objects \citep{RelationalDB}. 
This makes the database more storage efficient, but can make it slower than a non relational database, upon scaling.