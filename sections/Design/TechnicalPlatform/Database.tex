\subsection{Database}\label{ssc:tech_database}
As the system needed to be used on multiple different devices at the same time, a central storage control needed to be implemented. To ensure a reliable storage solution, a database has been chosen to act as the central storage system. Aalborg Zoo currently uses a Microsoft SQL Server 2012, this database server is an older version, which does not support JSON indexation (Allowing JSON elements to be searched as virtual columns \cite{MySQLJSON}). JSON indexation feature allowed for a more versatile system, as well making it possible to implement some of the requirements like "Function tags" \autoref{sc:requirements}, to be implemented at a later date. \\
Because JSON indexation was a requirement for the chosen database, MySQL was chosen as the database for the system. MySQL was chosen because MySQL is a free software, which contains the required functionality for developing the system. As well as its a well know SQL that has a large user base, which results in good first and third party documentation, so its easier for new developers to understand the database. \\

The database is configured as a relational database \cite{RelationalDB} so each element in that database is only present once, and any relation between objects are defined with relations between ID's. This makes the database more storage efficient, but can make it slower than a non relational database, upon scaling. \\
