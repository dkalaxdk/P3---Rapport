\chapter{Introduction}\label{ch:introduction}
As companies expand and acquire a growing number of assets, a desire to keep track of these assets and their condition grows equally. Creating an overview of the assets, and keeping this overview up-to-date, is a major task and one that companies sometimes do not address until after the number of assets has become overwhelming. Therefore a system managing the assets and their current state, would help keep an overview of the assets. The system could also include the ability to see where a specific asset is located, who currently has responsibility for it, and what department the asset belongs to.
\par
The concept of a system like this is not a new idea, as the problem has existed for years, but companies are very different and a "one-size-fits-all" static solution is not suitable for all. A way of making such a solution more suitable for a wider range of companies, could be to make it more dynamic. This gives the companies the opportunity to decide for themselves, what information is needed. With this approach, the individual company will not be overwhelmed with unneeded features, which are only relevant for a few specific companies.
\par
This project will be based on the needs of the IT department of Aalborg Zoo, and will focus on accommodating dynamicity and customizability.